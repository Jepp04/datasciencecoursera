% Options for packages loaded elsewhere
\PassOptionsToPackage{unicode}{hyperref}
\PassOptionsToPackage{hyphens}{url}
%
\documentclass[
]{article}
\usepackage{amsmath,amssymb}
\usepackage{lmodern}
\usepackage{ifxetex,ifluatex}
\ifnum 0\ifxetex 1\fi\ifluatex 1\fi=0 % if pdftex
  \usepackage[T1]{fontenc}
  \usepackage[utf8]{inputenc}
  \usepackage{textcomp} % provide euro and other symbols
\else % if luatex or xetex
  \usepackage{unicode-math}
  \defaultfontfeatures{Scale=MatchLowercase}
  \defaultfontfeatures[\rmfamily]{Ligatures=TeX,Scale=1}
\fi
% Use upquote if available, for straight quotes in verbatim environments
\IfFileExists{upquote.sty}{\usepackage{upquote}}{}
\IfFileExists{microtype.sty}{% use microtype if available
  \usepackage[]{microtype}
  \UseMicrotypeSet[protrusion]{basicmath} % disable protrusion for tt fonts
}{}
\makeatletter
\@ifundefined{KOMAClassName}{% if non-KOMA class
  \IfFileExists{parskip.sty}{%
    \usepackage{parskip}
  }{% else
    \setlength{\parindent}{0pt}
    \setlength{\parskip}{6pt plus 2pt minus 1pt}}
}{% if KOMA class
  \KOMAoptions{parskip=half}}
\makeatother
\usepackage{xcolor}
\IfFileExists{xurl.sty}{\usepackage{xurl}}{} % add URL line breaks if available
\IfFileExists{bookmark.sty}{\usepackage{bookmark}}{\usepackage{hyperref}}
\hypersetup{
  pdftitle={Exponential distribution in R vs the Central Limit Theorem},
  pdfauthor={Ulrich van Staden},
  hidelinks,
  pdfcreator={LaTeX via pandoc}}
\urlstyle{same} % disable monospaced font for URLs
\usepackage[margin=1in]{geometry}
\usepackage{color}
\usepackage{fancyvrb}
\newcommand{\VerbBar}{|}
\newcommand{\VERB}{\Verb[commandchars=\\\{\}]}
\DefineVerbatimEnvironment{Highlighting}{Verbatim}{commandchars=\\\{\}}
% Add ',fontsize=\small' for more characters per line
\usepackage{framed}
\definecolor{shadecolor}{RGB}{248,248,248}
\newenvironment{Shaded}{\begin{snugshade}}{\end{snugshade}}
\newcommand{\AlertTok}[1]{\textcolor[rgb]{0.94,0.16,0.16}{#1}}
\newcommand{\AnnotationTok}[1]{\textcolor[rgb]{0.56,0.35,0.01}{\textbf{\textit{#1}}}}
\newcommand{\AttributeTok}[1]{\textcolor[rgb]{0.77,0.63,0.00}{#1}}
\newcommand{\BaseNTok}[1]{\textcolor[rgb]{0.00,0.00,0.81}{#1}}
\newcommand{\BuiltInTok}[1]{#1}
\newcommand{\CharTok}[1]{\textcolor[rgb]{0.31,0.60,0.02}{#1}}
\newcommand{\CommentTok}[1]{\textcolor[rgb]{0.56,0.35,0.01}{\textit{#1}}}
\newcommand{\CommentVarTok}[1]{\textcolor[rgb]{0.56,0.35,0.01}{\textbf{\textit{#1}}}}
\newcommand{\ConstantTok}[1]{\textcolor[rgb]{0.00,0.00,0.00}{#1}}
\newcommand{\ControlFlowTok}[1]{\textcolor[rgb]{0.13,0.29,0.53}{\textbf{#1}}}
\newcommand{\DataTypeTok}[1]{\textcolor[rgb]{0.13,0.29,0.53}{#1}}
\newcommand{\DecValTok}[1]{\textcolor[rgb]{0.00,0.00,0.81}{#1}}
\newcommand{\DocumentationTok}[1]{\textcolor[rgb]{0.56,0.35,0.01}{\textbf{\textit{#1}}}}
\newcommand{\ErrorTok}[1]{\textcolor[rgb]{0.64,0.00,0.00}{\textbf{#1}}}
\newcommand{\ExtensionTok}[1]{#1}
\newcommand{\FloatTok}[1]{\textcolor[rgb]{0.00,0.00,0.81}{#1}}
\newcommand{\FunctionTok}[1]{\textcolor[rgb]{0.00,0.00,0.00}{#1}}
\newcommand{\ImportTok}[1]{#1}
\newcommand{\InformationTok}[1]{\textcolor[rgb]{0.56,0.35,0.01}{\textbf{\textit{#1}}}}
\newcommand{\KeywordTok}[1]{\textcolor[rgb]{0.13,0.29,0.53}{\textbf{#1}}}
\newcommand{\NormalTok}[1]{#1}
\newcommand{\OperatorTok}[1]{\textcolor[rgb]{0.81,0.36,0.00}{\textbf{#1}}}
\newcommand{\OtherTok}[1]{\textcolor[rgb]{0.56,0.35,0.01}{#1}}
\newcommand{\PreprocessorTok}[1]{\textcolor[rgb]{0.56,0.35,0.01}{\textit{#1}}}
\newcommand{\RegionMarkerTok}[1]{#1}
\newcommand{\SpecialCharTok}[1]{\textcolor[rgb]{0.00,0.00,0.00}{#1}}
\newcommand{\SpecialStringTok}[1]{\textcolor[rgb]{0.31,0.60,0.02}{#1}}
\newcommand{\StringTok}[1]{\textcolor[rgb]{0.31,0.60,0.02}{#1}}
\newcommand{\VariableTok}[1]{\textcolor[rgb]{0.00,0.00,0.00}{#1}}
\newcommand{\VerbatimStringTok}[1]{\textcolor[rgb]{0.31,0.60,0.02}{#1}}
\newcommand{\WarningTok}[1]{\textcolor[rgb]{0.56,0.35,0.01}{\textbf{\textit{#1}}}}
\usepackage{longtable,booktabs,array}
\usepackage{calc} % for calculating minipage widths
% Correct order of tables after \paragraph or \subparagraph
\usepackage{etoolbox}
\makeatletter
\patchcmd\longtable{\par}{\if@noskipsec\mbox{}\fi\par}{}{}
\makeatother
% Allow footnotes in longtable head/foot
\IfFileExists{footnotehyper.sty}{\usepackage{footnotehyper}}{\usepackage{footnote}}
\makesavenoteenv{longtable}
\usepackage{graphicx}
\makeatletter
\def\maxwidth{\ifdim\Gin@nat@width>\linewidth\linewidth\else\Gin@nat@width\fi}
\def\maxheight{\ifdim\Gin@nat@height>\textheight\textheight\else\Gin@nat@height\fi}
\makeatother
% Scale images if necessary, so that they will not overflow the page
% margins by default, and it is still possible to overwrite the defaults
% using explicit options in \includegraphics[width, height, ...]{}
\setkeys{Gin}{width=\maxwidth,height=\maxheight,keepaspectratio}
% Set default figure placement to htbp
\makeatletter
\def\fps@figure{htbp}
\makeatother
\setlength{\emergencystretch}{3em} % prevent overfull lines
\providecommand{\tightlist}{%
  \setlength{\itemsep}{0pt}\setlength{\parskip}{0pt}}
\setcounter{secnumdepth}{-\maxdimen} % remove section numbering
\ifluatex
  \usepackage{selnolig}  % disable illegal ligatures
\fi

\title{Exponential distribution in R vs the Central Limit Theorem}
\author{Ulrich van Staden}
\date{18/06/2021}

\begin{document}
\maketitle

\hypertarget{overview}{%
\subsection{Overview}\label{overview}}

This report demonstrates the differences and similarities between
theoretical and simulated results of an exponential distribution. The
results to be compared are the mean, variance and standard deviation.
First the exponential distribution was generated through using a 1000
random exponential values. Then the mean distribution were generated
through the use of averages of 40 random exponentials generated, of
which shows that the distribution tends to approximate a normal
distribution.

\hypertarget{simulations}{%
\subsection{Simulations}\label{simulations}}

The following theoretical parameters were given:

\begin{itemize}
\tightlist
\item
  \textbf{Rate} = lambda = 0.2
\item
  \textbf{Mean} = 1/lambda = 5
\item
  \textbf{Number of simulation} = 1000
\item
  \textbf{Number of exponentials} = 40
\end{itemize}

The parameters listed above were used to plot 2 graphs: an
\textbf{Exponential distribution} with a 1000 exponentials and a
\textbf{Mean distribution} generated through a 1000 simulations of
averages from 40 exponential distributions. These simulations have been
simulated in R programming.

\hypertarget{exponential-distribution}{%
\subsubsection{Exponential
Distribution}\label{exponential-distribution}}

\begin{Shaded}
\begin{Highlighting}[]
\NormalTok{n }\OtherTok{=} \DecValTok{1000} \CommentTok{\#Number of observations}
\NormalTok{lambda }\OtherTok{=} \FloatTok{0.2} \CommentTok{\# Rate parameter}
\NormalTok{Expo1000 }\OtherTok{=} \FunctionTok{rexp}\NormalTok{(n,lambda) }\CommentTok{\# Generates a 1000 random exponential values.}
\NormalTok{medianExp }\OtherTok{=} \FunctionTok{median}\NormalTok{(Expo1000)}
\NormalTok{meanExp }\OtherTok{=} \FunctionTok{mean}\NormalTok{(Expo1000)}
\CommentTok{\# Plot these values on a histogram.}
\FunctionTok{hist}\NormalTok{(Expo1000}
\NormalTok{     ,}\AttributeTok{main=}\StringTok{"Histogram on a 1000 Random Generated Exponetial Values"}
\NormalTok{     ,}\AttributeTok{xlab=}\StringTok{"Value"}
\NormalTok{     ,}\AttributeTok{breaks=}\DecValTok{1000}\NormalTok{)}
\FunctionTok{abline}\NormalTok{(}\AttributeTok{v =}\NormalTok{ medianExp, }\AttributeTok{col =} \StringTok{"blue"}\NormalTok{, }\AttributeTok{lwd =} \DecValTok{4}\NormalTok{) }\CommentTok{\# Draws a line presenting the median of the distribution.}
\FunctionTok{abline}\NormalTok{(}\AttributeTok{v =}\NormalTok{ meanExp, }\AttributeTok{col =} \StringTok{"red"}\NormalTok{, }\AttributeTok{lwd =} \DecValTok{4}\NormalTok{,) }\CommentTok{\# Draws a line presenting the mean of the distribution.}
\end{Highlighting}
\end{Shaded}

\includegraphics{test_files/figure-latex/unnamed-chunk-1-1.pdf}

The graph above demonstrates the frequency of the exponential values
generated. The red line represents the mean = 5.1261396 and the blue
line represents the median = 3.6913306, which indicates that half of the
generated values are below 3.5 and that most of the exponentials are
below approximately 5.

\hypertarget{mean-distribution}{%
\subsubsection{Mean Distribution}\label{mean-distribution}}

\begin{Shaded}
\begin{Highlighting}[]
\NormalTok{n}\OtherTok{\textless{}{-}}\DecValTok{40} \CommentTok{\#Number of observations}
\NormalTok{lambda }\OtherTok{\textless{}{-}} \FloatTok{0.2} \CommentTok{\# Rate parameter}
\NormalTok{Expo1000 }\OtherTok{=} \ConstantTok{NULL}

\CommentTok{\# Generates a 1000 random averages from 40 random generated exponential values.}
\ControlFlowTok{for}\NormalTok{ (i }\ControlFlowTok{in} \DecValTok{1} \SpecialCharTok{:} \DecValTok{1000}\NormalTok{) Expo1000 }\OtherTok{=} \FunctionTok{c}\NormalTok{(Expo1000, }\FunctionTok{mean}\NormalTok{(}\FunctionTok{rexp}\NormalTok{(n,lambda)))}
\NormalTok{medianExp }\OtherTok{=} \FunctionTok{median}\NormalTok{(Expo1000)}
\NormalTok{meanExp }\OtherTok{=} \FunctionTok{mean}\NormalTok{(Expo1000)}
\FunctionTok{hist}\NormalTok{(Expo1000}
\NormalTok{     ,}\AttributeTok{main=}\StringTok{"Histogram on a 1000 Simulations on Averages of 40 Exponetial Values"}
\NormalTok{     ,}\AttributeTok{xlab=}\StringTok{"Average/Mean"}
\NormalTok{     ,}\AttributeTok{breaks=}\DecValTok{40}\NormalTok{)}
\FunctionTok{abline}\NormalTok{(}\AttributeTok{v =}\NormalTok{ medianExp, }\AttributeTok{col =} \StringTok{"blue"}\NormalTok{, }\AttributeTok{lwd =} \DecValTok{4}\NormalTok{) }\CommentTok{\# Draws a line presenting the median of the distribution.}
\FunctionTok{abline}\NormalTok{(}\AttributeTok{v =}\NormalTok{ meanExp, }\AttributeTok{col =} \StringTok{"red"}\NormalTok{, }\AttributeTok{lwd =} \DecValTok{4}\NormalTok{,) }\CommentTok{\# Draws a line presenting the mean of the distribution.}
\end{Highlighting}
\end{Shaded}

\includegraphics{test_files/figure-latex/unnamed-chunk-2-1.pdf}

The graph above demonstrates the mean frequency of the random
exponential values generated. The red line represents the mean =
4.9974042 and the blue line represents the median = 4.9577324. As you
can see the median and the mean of this distribution are quite close and
this is due to the average taken of from every 40 exponential values
generated.

\hypertarget{sample-mean-versus-theoretical-mean}{%
\subsection{Sample Mean versus Theoretical
Mean}\label{sample-mean-versus-theoretical-mean}}

\begin{Shaded}
\begin{Highlighting}[]
\CommentTok{\# Sample mean of the simulation}
\NormalTok{SampleMean }\OtherTok{=} \FunctionTok{mean}\NormalTok{(Expo1000)}
\CommentTok{\#Theoretical Mean of the simulation}
\NormalTok{TheoreticalMean }\OtherTok{=} \DecValTok{1}\SpecialCharTok{/}\NormalTok{lambda}
\CommentTok{\#Confidence Interval of the simulation, base on 95\% confidence}
\NormalTok{ConfidenceInterval }\OtherTok{=} \FunctionTok{t.test}\NormalTok{(Expo1000)}\SpecialCharTok{$}\NormalTok{conf.int}
\end{Highlighting}
\end{Shaded}

\begin{longtable}[]{@{}lccc@{}}
\toprule
Parameters & Simulation & Theoretical & Confidence Interval \\
\midrule
\endhead
Mean & 4.9974042 & 5 & 4.9485661, 5.0462423 \\
\bottomrule
\end{longtable}

The mean is quite similar between the theoretical mean and the simulated
mean and the theoretical mean is contained within the confidence
interval.

\hypertarget{sample-variance-versus-theoretical-variance}{%
\subsection{Sample Variance versus Theoretical
Variance}\label{sample-variance-versus-theoretical-variance}}

\begin{Shaded}
\begin{Highlighting}[]
\CommentTok{\# Sample Standard Deviation of the simulation}
\NormalTok{SampleStadard }\OtherTok{=} \FunctionTok{sd}\NormalTok{(Expo1000)}
\CommentTok{\#Theoretical Standard Deviation of the simulation}
\NormalTok{TheoreticalStandard }\OtherTok{=}\NormalTok{ (}\DecValTok{1}\SpecialCharTok{/}\NormalTok{lambda)}\SpecialCharTok{/}\FunctionTok{sqrt}\NormalTok{(n)}
\CommentTok{\# Sample Variance of the simulation}
\NormalTok{SampleVar }\OtherTok{=} \FunctionTok{var}\NormalTok{(Expo1000)}
\CommentTok{\#Theoretical Variance of the simulation}
\NormalTok{TheoreticalVar }\OtherTok{=}\NormalTok{ (}\DecValTok{1}\SpecialCharTok{/}\NormalTok{lambda)}\SpecialCharTok{\^{}}\DecValTok{2}\SpecialCharTok{/}\NormalTok{n}
\end{Highlighting}
\end{Shaded}

\begin{longtable}[]{@{}lcc@{}}
\toprule
Parameters & Simulation & Theoretical \\
\midrule
\endhead
Standard Deviation & 0.7870168 & 0.7905694 \\
Variance & 0.6193955 & 0.625 \\
\bottomrule
\end{longtable}

Both the standard deviation and the variance are quite similar between
the theoretical results and the simulated results. \#\#\# Distribution

\begin{Shaded}
\begin{Highlighting}[]
\CommentTok{\# Plots the mean distribution.}
\FunctionTok{hist}\NormalTok{(Expo1000}
\NormalTok{     ,}\AttributeTok{main=}\StringTok{"Histogram on a 1000 Simulations on Averages of 40 Exponetial Values"}
\NormalTok{     ,}\AttributeTok{xlab=}\StringTok{"Average/Mean"}
\NormalTok{     ,}\AttributeTok{prob=}\ConstantTok{TRUE}
\NormalTok{     ,}\AttributeTok{breaks =} \DecValTok{40}\NormalTok{)}
\FunctionTok{lines}\NormalTok{(}\FunctionTok{density}\NormalTok{(Expo1000), }\AttributeTok{col =} \StringTok{"green"}\NormalTok{, }\AttributeTok{lwd =} \DecValTok{4}\NormalTok{) }\CommentTok{\# Plots a line tracing an aproximation to a normal distribution.}
\FunctionTok{abline}\NormalTok{(}\AttributeTok{v =}\NormalTok{ medianExp, }\AttributeTok{col =} \StringTok{"blue"}\NormalTok{, }\AttributeTok{lwd =} \DecValTok{4}\NormalTok{) }\CommentTok{\# Draws a line presenting the median of the distribution.}
\FunctionTok{abline}\NormalTok{(}\AttributeTok{v =}\NormalTok{ meanExp, }\AttributeTok{col =} \StringTok{"red"}\NormalTok{, }\AttributeTok{lwd =} \DecValTok{4}\NormalTok{,) }\CommentTok{\# Draws a line presenting the mean of the distribution.}
\end{Highlighting}
\end{Shaded}

\includegraphics{test_files/figure-latex/unnamed-chunk-5-1.pdf}

The same mean distribution is plotted above as in the simulations
section with one extra line (green) indicating the approximation to a
normal distribution. As the number of samples or simulation increase so
will the normaility of the distribution become more normal.

\end{document}
